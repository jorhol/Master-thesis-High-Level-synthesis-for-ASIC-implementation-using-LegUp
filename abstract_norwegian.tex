\pagestyle{empty}
\renewcommand{\abstractname}{Sammendrag}
\begin{abstract}
Med den økende etterspørselen etter lavt effektforbruk og lite areal i store System-på-chip design, bestående av milliarder av transistorer, er ikke lenger den typiske design-metoden brukende dersom maskinvare-produsentene ønsker å tilby det beste produktet på markedet.

Arkitektur-utforsking er en viktig del av designprosessen, hvor flere design skapes og blir evaluert i form av areal, ytelse, og effektforbruk. Høy-nivå syntese (HLS) er et attraktivt konsept for å redusere den samlede innsatsen designeren må legge ned i arkitektur-utforskingen. Ved å benytte HLS i et rammeverk for arkitektur-utforsking av digital maskinvare kan antallet arkitekturelle variasjoner som kan genereres og evalueres økes drastisk sammenlignet med å utføre arbeidet manuelt.

I et tidligere prosjekt ble HLS-verktøyet \textit{LegUp} utforsket. Målet var å undersøke om verktøyet kunne brukes i et slikt rammeverk. Konklusjonen fra prosjektet var at noen problemer med LegUp begrenset muligheten til å generere Register-Transfer Level (RTL)-kode egnet til implementering på applikasjonsspesifikk integrert krets (ASIC) arkitekturer.

Denne avhandling presenterer en løsning for et rammeverk for arkitek-tur-utforskring bygget på en tilpasset versjon av LegUp. Rammeverket kan generere et stort antall arkitekturelle variasjoner av et design skrevet i C, og kjøre simulering, syntese, layout og effekt analyse på hvert design. Randomiserte føringer benyttes i rammeverket for å generere varierte design fra HLS-verktøyet. Rammeverket genererer rapporter som beskriver arealbruk, maksimal ytelse, og beregnet effektforbruk for hvert design, slik at designeren kan velge det designet som passer best, basert på avveininger mellom viktige parametre fra designspesifikasjonen.

Et \textit{konseptbevis} ble utført ved å kjøre et FIR-filter design gjennom rammeverket. Resultatet viste at en besparelse i areal på 13.28\% og en besparelse i effektforbruk på 9.52\% kunne oppnås ved å velge det designet med best resultater over designet med dårligst resultater. Disse resultatene viser at konseptet fungerer. HLS-verktøyet genererer en økning i areal og effektforbruk sammenlignet med et tilsvarende design skrevet direkte i RTL-kode på mellom 30-200\%, noe som gjør det lite økonomisk å benytte verktøyet til design av maskinvare. Forholdet mellom de genererte resultatene ser likevel ut til å stemme (høy \textit{fidelity}), noe som gjør at rammeverk-resultatene kan benyttes til å velge arkitektur for designet. Gjennom prosessen med å tilpasse LegUp til å generer kode som støttes av verktøyene i rammeverket har noe av den originale funksjonaliteten gått tapt. Noen feil har også oppstått og blitt oppdaget. Før rammeverket brukes til noen form for kommersielle formal må alle problemer som er beskrevet i denne rapporten elimineres.
\end{abstract}