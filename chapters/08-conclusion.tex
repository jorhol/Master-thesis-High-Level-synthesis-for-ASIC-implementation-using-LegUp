\chapter{Conclusion}
\label{chp:conclusion} 
This thesis has presented a solution for adapting the \gls{hls}-tool LegUp, to make the tool produce Verilog output suitable for synthesis towards ASIC architectures. In addition to this, a framework for architectural exploration of digital hardware has been developed, using LegUp to increase the abstraction-level of a functional description. The framework is capable of generating a wide variety of architectural implementations of the given functional specification, using randomized constraints in the \gls{hls}-flow to get varying output. Each of the designs generated by \gls{hls} will automatically be simulated and synthesized, before layout and power analysis will be performed on the design. The framework will generate reports, containing interesting factors like area, power estimates, speed, and register count for each design. The reports allow for the designer to easily compare the designs and select the one best suited based on the specification.

To ensure the functionality of the framework, a proof-of-concept has been conducted, using a FIR-filter as the reference design. The results from the framework shows that three of the six randomized constraint used, had an impact on the generated design. This gave a total of 8 architectural variations. By comparing the best and worst of the generated architectural variations, a decrease in area of 13.28\% and a decrease in power consumption of 9.52\% could be achieved. These results indicate that the proof-of-concept works.

When running the framework, some problems and bugs in the generated design were observed. Some of these were related to how the adaption of LegUp is performed, while some could be bugs in LegUp or the tool-flow. It is also an element of unease that the way you write your functional specification can affect the generated designs greatly. The compiler or LegUp should be able to optimize away any parts of the code that is obsolete. If this concept are to be used at a professional level, it is vital that all parts of the framework generate error-free results. 

The final conclusion is that the concept works and that LegUp, in the adapted version described in this thesis, can be used in a framework for architectural exploration. However, much more effort should be put into verification and testing of all parts of the tool-flow, to ensure all bugs and potential errors are eliminated before using it for any commercial purpose.
\section{Future work}
\label{sec:futurework} 
\subsection{Abstraction level}
During the adaption of LegUp to make the generated Verilog more suitable for ASIC implementation, much of the higher level of abstraction found in ANSI-C that were originally supported by LegUp, is lost. This limitation on how the code can be written reduces the usefulness of the framework. If a better solution to the resolved issues can be found, or some other solution that brings back the desired functionality from ANSI-C, this would increase the value of the created tool greatly.

\subsection{Resolving bugs}
In \cref{sec:designbugs} two bugs that were noticed in simulation have been described. For the framework to be useful, the generated designs should be bug-free, given that no bugs are present in the functional description. It should be put some effort into figuring out what is creating these bugs, and find a solution to avoid that these, and other bugs are generated in future designs.

\subsection{Automatic code-optimization}
The overhead in both area and power-consumption could be reduced greatly by manually optimizing the code input to the framework, as described in \cref{sec:codeoptimization}. This is a huge drawback of the framework, as it forces the designer to keep focus on writing the code in the most correct way, instead of focusing on writing the correct functional specification. The task of optimizing the input should be left to the compiler or \gls{hls}-tool. It would be strongly beneficial if a solution could be implemented to ensure that the code is optimized correctly.

\subsection{Incorporating Nordic Semiconductors DDVC}
The last objective described in \cref{sec:projectobj} have not been considered in this thesis, as the work with providing a functional framework and creating the proof-of-concept took all of the available time. It would still be interesting to see if the incorporation of Nordic Semiconductors \gls{ddvc} into the Verilog-generating libraries of LegUp will reduce the overhead of the tool.s