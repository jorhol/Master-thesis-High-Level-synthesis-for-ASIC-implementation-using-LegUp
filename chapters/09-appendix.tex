\chapter{Appendix}
\section{\label{sec:sourcecode}Source code of FIR-filter reference design}
\subsection{\label{subsec:cfircode}C source code}
\lstset{language=C,style=CStyle}
\begin{lstlisting}
#include <stdint.h>

#define INPUTSIZE 32
#define TAPS 16

int main(int done, int inData, volatile long long int __out_outData) {

  int sr[TAPS-1] = {0};
  long long int products[TAPS] = {0};
  long long int sum = 0;
  while(done == 0){
    products[0] = inData * 1;
    sum = products[0];
    for (int i = 1; i < TAPS ; i++){
      for(int j = TAPS-1; j > 0; j--){
        sr[j] = sr[j -1];
      }
      sr[0] = inData;
      for (int k = 1; k < TAPS ; k++){
        products[k] = sr[k-1] * (k+1);
      }
      sum = sum + products[i];
    }
    __out_outData = sum;
  }
  return 0;
}
\end{lstlisting}

\subsection{\label{subsec:cfircode2}Otimized C source code}
#include <stdint.h>

#define INPUTSIZE 32
#define TAPS 16

int main(int done, int inData, volatile long long int __out_outData) {
  int sr[TAPS-1] = {0};
  long long int sum = 0;
  while(done == 0){
    sum = inData;
    for (int i = 1; i < TAPS ; i++){
      for(int j = TAPS-1; j > 0; j--){
        sr[j] = sr[j -1];
      }
      sr[0] = inData;
      sum = sum + sr[i-1] * (i+1);
    }
    __out_outData = sum;
  }
  return 0;
}

\subsection{\label{subsec:cfircode}Verilog source code}
\lstset{language=Verilog,style=VerilogStyle}
\begin{lstlisting}
module fir (
    clk,
    reset,
    dataIn,
    dataOut
  );
  parameter WIDTH = 32;
  parameter DEPTH = 16;

		
  input clk, reset;
  input signed [WIDTH-1:0] dataIn;
  output wire signed [2*WIDTH-1:0] dataOut;
  integer i, j, k;
  reg signed [WIDTH-1:0] sr [DEPTH-2:0];
  reg signed [2*WIDTH-1:0] products [DEPTH-1:0];
  reg signed [2*WIDTH-1:0] sum;

  always @( posedge clk or posedge reset ) begin
    if( reset == 1) begin
      sum = 0;
    end
    else begin
	  for(i = DEPTH-2; i > 0; i=i-1) begin
        sr[i] <= sr[i-1];
      end
	  sr[0] <= dataIn;
    end
  end

  always @(*) begin
	products[0] = dataIn * 1;
    for (j = 1; j < DEPTH ; j=j+1) begin
      products[j] = sr[j-1] * (j+1);
	end
  end

  always @(*) begin
    sum = products [0];
    for (k = 1; k < DEPTH ; k=k+1) begin
      sum = sum + products[k];
	end
  end
	
  assign dataOut = sum;
	
endmodule
\end{lstlisting}

\subsection{\label{subsec:firfiltertb}Testbench for FIR-filter}
\lstset{language=Verilog,style=VerilogStyle}
\begin{lstlisting}
`timescale 1 ns / 1 ns
module test_fir
(
);

reg  clk;
reg  reset;
reg  start;
wire [63:0] return_val;
wire  finish;
reg  memory_controller_waitrequest;
reg [31:0] arg_done;
reg [31:0] arg_inData;
wire [63:0] arg_outData;
wire  arg_outData_valid;
wire  iterationFinish;


fir u_fir (
	.clk (clk),
	.reset (reset),
	.start (start),
	.finish (finish),
	.memory_controller_waitrequest (memory_controller_waitrequest),
	.return_val (return_val),
	.arg_done (arg_done),
	.arg_inData (arg_inData),
	.arg_outData (arg_outData),
	.arg_outData_valid (arg_outData_valid),
	.iterationFinish (iterationFinish)
);

// Clock generation
initial 
    clk = 0;
always @(clk)
    clk <= #10 ~clk;

initial begin
@(negedge clk);
reset <= 1;
@(negedge clk);
reset <= 0;
start <= 1;
@(negedge clk);
start <= 0;
end

always@(finish) begin
    if (finish == 1) begin
        $display("At t=%t simulation finished", $time);
        $display("Cycles: %d", ($time-50)/20);
        $finish;
    end
end

initial begin
memory_controller_waitrequest <= 1;
@(negedge clk);
@(negedge clk);
memory_controller_waitrequest <= 0;
end

// Custom testcases:     
  initial begin
    arg_inData = 32'b0; 
		arg_done = 32'b0;
  end 
	
	initial  
  begin : TEST_CASE 
		@(posedge reset)
		repeat (1000) begin 
			@(negedge clk); 
			arg_inData = $random; 
			@(posedge iterationFinish); 
		end	
		$display("Finished applying inData\n");
		arg_done = 32'b1;
		#100ns
		$finish;
  end 

endmodule 
\end{lstlisting}