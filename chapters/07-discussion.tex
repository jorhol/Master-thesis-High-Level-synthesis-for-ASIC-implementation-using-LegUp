\chapter{Discussion}

By comparing the best-case design towards the worst-case design, a potential area saving of 50872.8523 and power sawing of 0.670 mW is achieved. This corresponds to a decrease in area of 13.28\% and a decrease in power consumption of 9.52\%. Compared to the design written directly in Verilog, the best-case area is 282.69\% and power consumption is 331.34\% of the results achieved in the Verilog design. An overhead of almost 200\% is not a great result, but the idea here is not to get a comparable result rather to show that the concept works and can be used for a framework for architectural exploration. This goal has been achieved, as we get varied outputs depending on the given HLS constraints.

\section{Advances in LegUp since last release}
From the GIT repository of LegUp, it can be seen that many new features have been and still are being implemented for the next release version of LegUp. The current version (4.0), were released in August 2015. One of the more interesting features from the views of this project is the implementation of streaming inputs and outputs. Even though this has been implemented in this project as well, the native implementation can be more thoroughly tested and have more functionality than the one implemented here. However, it is still uncertain if a way of producing multiple output-signals have been implemented.

http://lists.legup.org/pipermail/legup-commits/2015-September/002603.html

commit 40dbecc7f39a5d8c6c897497a4b4b5ddef3da3a0
Author: Jongsok Choi <jongsok.choi at gmail.com>
Date:   Fri Sep 25 20:08:53 2015 -0400

    Added member variable "isExternalRAM" to indicate if a RAM exists
    outside the circuit and is not to be created inside (say for a pointer
    that is passed in as an argument into the top-level function).
    
