\chapter{Creating the Framework}
\section{Create New Project}
To create a new project, the directory hierarchy needs to be copied from a source and directories and filenames needs to be altered to match the design name. Filelist and setting files also needs to be altered to include the correct design file names. This process is automated into a bash script, \textit{CreateNewProject.sh}. To run the script, the directory \textit{\_source}, containing the source project, must be present in the directory where the script are called.

The directory and file tree of the framework is shown in \cref{fig:frameworkdirtree}. Directories are colored cyan, executables and scripts are colored teal, and other design and constraint files are colored violet. File comment or description are in black. Each file is described in the comment on the right side. 
\begin{figure}
\centering

\begin{minipage}{0.99\textwidth}
    \renewcommand*\DTstyle{\sffamily\tiny}
    \dirtree{%
    .1 \textcolor{cyan}{\slash}.
    .2 \textcolor{cyan}{\_source}.
    .3 \textcolor{cyan}{ip}.
    .4 \textcolor{cyan}{designname}.
    .5 \textcolor{cyan}{hls}.
    .6 \textcolor{teal}{constraintsGenerator{.}run} \dotfill \:\:\begin{minipage}[t]{5.4cm}
                                                    Program to generate constraint and Makefiles
                                                    \end{minipage}.
    .6 \textcolor{violet}{constraints{.}xlsx} \dotfill \:\:\begin{minipage}[t]{5.4cm}
                                                            Setup-file for constraint-generator
                                                            \end{minipage}. 
    .5 \textcolor{cyan}{lay}.
    .6 \textcolor{teal}{Makefile} \dotfill \:\:\begin{minipage}[t]{5.4cm}
                                                    Makefile for running layout
                                                    \end{minipage}.
    .5 \textcolor{cyan}{pow}.
    .6 \textcolor{teal}{Makefile} \dotfill \:\:\begin{minipage}[t]{5.4cm}
                                                    Makefile for running power analysis
                                                    \end{minipage}.
    .6 \textcolor{teal}{power\_analysis{.}tcl} \dotfill \:\:\begin{minipage}[t]{5.4cm}
                                                    Settings file for power analysis
                                                    \end{minipage}.
    .5 \textcolor{cyan}{rtl}.
    .6 \textcolor{violet}{designname{.}fl} \dotfill \:\:\begin{minipage}[t]{5.4cm}
                                                    Filelist specifying files in the design for synthesis
                                                    \end{minipage}.
    .6 \textcolor{violet}{designname{.}v} \dotfill \:\:\begin{minipage}[t]{5.4cm}
                                                    Verilog designfile generated by LegUp
                                                    \end{minipage}.
    .5 \textcolor{cyan}{sim}.
    .6 \textcolor{cyan}{run}.
    .7 \textcolor{violet}{designname{.}args} \dotfill \:\:\begin{minipage}[t]{5.4cm}
                                                            Argument simulation tool
                                                            \end{minipage}.
    .7 \textcolor{violet}{designname{.}comp} \dotfill \:\:\begin{minipage}[t]{5.4cm}
                                                            Compilation parameter file for simulation{.} Specifies filelist for design and testbench{.}
                                                            \end{minipage}.
    .7 \textcolor{violet}{designname{.}sim} \dotfill \:\:\begin{minipage}[t]{5.4cm}
                                                            Simulation parameter file for simulation{.} Specifies top-level testbench module and simulation options{.}
                                                            \end{minipage}.
    .7 \textcolor{violet}{modelsim{.}ini} \dotfill \:\:\begin{minipage}[t]{5.4cm}
                                                            Settings-file for simulation tool
                                                            \end{minipage}.
    .7 \textcolor{teal}{RUN\_ALL} \dotfill \:\:\begin{minipage}[t]{5.4cm}
                                                            Script to run simulation
                                                            \end{minipage}.
    .6 \textcolor{cyan}{tb}.
    .7 \textcolor{violet}{test\_designname{.}fl} \dotfill \:\:\begin{minipage}[t]{5.4cm}
                                                    Filelist specifying files in testbench
                                                    \end{minipage}.
    .7 \textcolor{violet}{test\_designname{.}v} \dotfill \:\:\begin{minipage}[t]{6cm}
                                                    Verilog testbench file generated by LegUp
                                                    \end{minipage}.
    .7 \textcolor{violet}{test\_designname\_testcases{.}v} \dotfill \:\:\begin{minipage}[t]{6cm}
                                                    File containing testcases to be included in testbench generation in LegUp
                                                    \end{minipage}.
    .5 \textcolor{cyan}{syn}.
    .6 \textcolor{cyan}{dc\_scripts}.
    .7 \textcolor{violet}{designname{.}constraints{.}tcl} \dotfill \:\:\begin{minipage}[t]{5.4cm}
                                                            Synthesis constraint file{.} Clocks are specified here{.}
                                                            \end{minipage}.
    .6 \textcolor{violet}{common\_setup{.}tcl} \dotfill \:\:\begin{minipage}[t]{5.4cm}
                                                            Setup file for synthesis 
                                                            \end{minipage}.
    .6 \textcolor{teal}{Makefile} \dotfill \:\:\begin{minipage}[t]{5.4cm}
                                                            Makefile for running synthesis
                                                            \end{minipage}.
    .5 \textcolor{violet}{designname{.}c} \dotfill \:\:\begin{minipage}[t]{5.4cm}
                                                            C file for functions design specification
                                                            \end{minipage}.
    .5 \textcolor{teal}{HLSScript{.}sh} \dotfill \:\:\begin{minipage}[t]{5.4cm}
                                                            Script for running framework
                                                            \end{minipage}. 
    .5 \textcolor{teal}{Makefile} \dotfill \:\:\begin{minipage}[t]{5.4cm}
                                                            Makefile for running tool-flow without framework/HLS
                                                            \end{minipage}. 
    .5 \textcolor{violet}{results{.}xlsx} \dotfill \:\:\begin{minipage}[t]{5.4cm}
                                                            File to visualize and compare framework results
                                                            \end{minipage}. 
    .4 \textcolor{cyan}{libs}.
    .5 \textcolor{cyan}{altera}.
    .6 \textcolor{violet}{altera\_mf{.}v} \dotfill \:\:\begin{minipage}[t]{5.4cm}
                                                            Library file containing help-function for memory operations, used in simulation
                                                            \end{minipage}. 
    .3 \textcolor{cyan}{methodology} \dotfill \:\:\begin{minipage}[t]{5.4cm}
                                                            Scripts and utilities for toolchain
                                                            \end{minipage}. 
    .2 \textcolor{teal}{CreateNewProject{.}sh} \dotfill \:\:\begin{minipage}[t]{5.4cm}
                                                            Script for creating new project
                                                            \end{minipage}. 
    }
\end{minipage}
\caption{Directory and file-tree of the framework}
\label{fig:frameworkdirtree}
\end{figure}

\section{HLS-script}
In order to automatically run HLS, simulation, synthesis, layout, and power-estimation, a script is created. Since LegUp is running on a VirtualBox image and not on the same servers where the other tools are located, some files and commands needs to be transferred between different locations. In the project \cite{holm2015pro} a possible solution using SSH and SCP were proposed. The script will build on this method, but first some additional preparations needs to be made. Since the VirtualBox guest is running on a local computer, a port forwarding rule has to be added to allow connections to port 22 of the guest from the Nordic Linux servers. The connections has to go through the host computer, as the VirtualBox guest does not have any direct connection to the network. The setting can either be set using the GUI of VirtualBox, or by running the following command from a command line:
\begin{verbatim}
  VBoxManage modifyvm myserver --natpf1 "ssh,tcp,,3022,,22"  
\end{verbatim}
Here "myserver" is the name of the VirtualBox VM and should be replaced with whatever you called the VM when you created it in VirtualBox. When this setting is set, it is possible to establish a connection over SSH from the Linux server directly to LegUp by using the port 3022 and the local IP-address of the computer running VirtualBox.

The standard SSH and SCP packages on Linux systems does not support passing the password as an argument to the command. To avoid the need to enter username and password each time a file is transferred using SCP or a command is executed using SSH, it is necessary to setup key-based authentication. This can be done manually, but the framework-script can also do this setup automatically if you pass the flag \textit{-s}. What the script does, is to generate a RSA key-file for the current user with \textit{ssh-keygen} and copy the generated file to LegUp using \textit{ssh-copy-id}. The password is passed to \textit{ssh-copy-id} using \textit{spawn}, \textit{expect} and \textit{send} commands.

The script has 7 main tasks:
\begin{itemize}
    \item Generate constraint and Makefiles
    \item Run HLS-tool to generate Verilog design file
    \item Run simulation
    \item Run synthesis
    \item Run layout
    \item Run power estimation
    \item Generate readable result reports
\end{itemize}
The five tasks in the middle is basically just copying files to and from LegUp, and running make-commands. These steps are a combination of the standard tool-flow used at Nordic semiconductor, part of the tool-flow created by Joar Talstad in his Master-thesis, and some necessary additions. The two last tasks however, are a bit more comprehensive and are described in the following two subsections. The script is written for the bash-shell, and the full code of the script is listen in \cref{lst:hlsscript} in \cref{sec:hlsscript}.

\subsection{Constraint generating}
In order to run HLS with a variety of different constraints and settings, one constraint and one Makefile need to be generated for each run. To automate this process, an Excel document, \textit{constraints.xlsx}, has been created. Sheet 2 of this document contains the setup of the constraints. Here the user can select which constraints should be randomized and also set if the constraint should have a specific value. Some values are required, and must have the value specified. If a parameter is not needed, the user can specify that the default value of the parameter should be kept. 
Sheet 1 of the Excel file contains a comma-separated version (CSV format) of the constraint settings. This CSV format is copied to a CSV file by using the headless tool \textit{convert-to} in libreoffice, \verb!libreoffice --headless --convert-to csv!. The CSV-file can then be read by the program generating constraint- and Makefiles. The program takes the filename of the CSV-file, the level parameter for the Makefile, and the designname as inputs, and returns the number of generated constraint-files. This number is used in the HLS-script to run the framework the correct number of times. The full code of the program generating constraint files are listed in \cref{lst:constraintGenerating} in \cref{sec:hlsscript}.

\subsection{Report generating}
As the HLS-script can be used to generate a large amount of designs, it is important to easily be able to collect the important data from all the generated reports. To ease the process of data collecting, the script collects the data from all designs and stores it in separate files. The data is collected using \textit{grep} commands, and the data is stripped of unnecessary text using \textit{find} and \textit{sed} commands. The collected data is stored under the folder reports, with the filenames:
\begin{verbatim}
net_switching_power.rpt     
cell_internal_power.rpt     
cell_leakage_power.rpt      
total_power.rpt          

register_count.rpt
comb_area.rpt
noncomb_area.rpt
buf_inv_area.rpt
cell_area.rpt
\end{verbatim}

The content of the files can easily be imported in Excel or other tools for creating graphs or do comparison. This comparison could be made automatic if desired.