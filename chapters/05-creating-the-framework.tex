\chapter{Creating the Framework}
\section{Create New Project}
To create a new project, the directory hierarchy needs to be copied from a source and directories and filenames needs to be altered to match the design name. Filelist and setting files also needs to be altered to include the correct design file names. This process is automated into a bash script, \textit{CreateNewProject.sh}. To run the script, the directory \textit{\_source}, containing the source project, must be present in the directory where the script are called.

The directory and file tree of the framework is shown in \cref{fig:frameworkdirtree}. Directories are colored cyan, executables and scripts are colored teal, and other design and constraint files are colored violet. File comment or description are in black.
\begin{figure}
\centering

\begin{minipage}{0.99\textwidth}
    \renewcommand*\DTstyle{\ttfamily\scriptsize}
    \dirtree{%
    .1 \textcolor{cyan}{\slash}.
    .2 \textcolor{cyan}{\_source}.
    .3 \textcolor{cyan}{ip}.
    .4 \textcolor{cyan}{designname}.
    .5 \textcolor{cyan}{hls}.
    .6 \textcolor{teal}{constraintsGenerator{.}run} \dotfill \:\:\begin{minipage}[t]{6cm}
                                                    Program to generate constraint and Makefiles
                                                    \end{minipage}.
    .6 \textcolor{violet}{constraints{.}xlsx} \dotfill \:\:\begin{minipage}[t]{6cm}
                                                            Setup-file for constraint-generator
                                                            \end{minipage}. 
    .5 \textcolor{cyan}{rtl}.
    .6 \textcolor{violet}{designname{.}fl} \dotfill \:\:\begin{minipage}[t]{6cm}
                                                    Filelist specifying files in the design for synthesis
                                                    \end{minipage}.
    .6 \textcolor{violet}{designname\_sim{.}fl} \dotfill \:\:\begin{minipage}[t]{6cm}
                                                    Filelist specifying files in the design for simulation{.} Altera libraries required for simulation is not synthesisable{.}
                                                    \end{minipage}.
    .6 \textcolor{violet}{designname{.}v} \dotfill \:\:\begin{minipage}[t]{6cm}
                                                    Verilog designfile generated by LegUp
                                                    \end{minipage}.
    .5 \textcolor{cyan}{sim}.
    .6 \textcolor{cyan}{run}.
    .7 \textcolor{violet}{designname{.}args} \dotfill \:\:\begin{minipage}[t]{6cm}
                                                            Argument simulation tool
                                                            \end{minipage}.
    .7 \textcolor{violet}{designname{.}comp} \dotfill \:\:\begin{minipage}[t]{6cm}
                                                            Compilation parameter file for simulation{.} Specifies filelist for design and testbench{.}
                                                            \end{minipage}.
    .7 \textcolor{violet}{designname{.}sim} \dotfill \:\:\begin{minipage}[t]{6cm}
                                                            Simulation parameter file for simulation{.} Specifies top-level testbench module and simulation options{.}
                                                            \end{minipage}.
    .7 \textcolor{violet}{modelsim{.}ini} \dotfill \:\:\begin{minipage}[t]{6cm}
                                                            Settings-file for simulation tool
                                                            \end{minipage}.
    .7 \textcolor{teal}{RUN\_ALL} \dotfill \:\:\begin{minipage}[t]{6cm}
                                                            Script to run simulation
                                                            \end{minipage}.
    .6 \textcolor{cyan}{tb}.
    .7 \textcolor{violet}{test\_designname{.}fl} \dotfill \:\:\begin{minipage}[t]{6cm}
                                                    Filelist specifying files in testbench
                                                    \end{minipage}.
    .7 \textcolor{violet}{test\_designname{.}v} \dotfill \:\:\begin{minipage}[t]{6cm}
                                                    Verilog testbench file generated by LegUp
                                                    \end{minipage}.
    .7 \textcolor{violet}{test\_designname\_testcases{.}v} \dotfill \:\:\begin{minipage}[t]{6cm}
                                                    File containing testcases to be included in testbench generation in LegUp
                                                    \end{minipage}.
    .5 \textcolor{cyan}{syn}.
    .6 \textcolor{cyan}{dc\_scripts}.
    .7 \textcolor{violet}{designname{.}constraints{.}tcl} \dotfill \:\:\begin{minipage}[t]{6cm}
                                                            Synthesis constraint file{.} Clocks are specified here{.}
                                                            \end{minipage}.
    .6 \textcolor{violet}{common\_setup{.}tcl} \dotfill \:\:\begin{minipage}[t]{6cm}
                                                            Setup file for synthesis 
                                                            \end{minipage}.
    .6 \textcolor{teal}{Makefile} \dotfill \:\:\begin{minipage}[t]{6cm}
                                                            Makefile for running synthesis
                                                            \end{minipage}.
    .5 \textcolor{violet}{designname{.}c} \dotfill \:\:\begin{minipage}[t]{6cm}
                                                            C file for functions design specification
                                                            \end{minipage}.
    .5 \textcolor{teal}{HLSScript{.}sh} \dotfill \:\:\begin{minipage}[t]{6cm}
                                                            Script for running framework
                                                            \end{minipage}. 
    .5 \textcolor{violet}{results{.}xlsx} \dotfill \:\:\begin{minipage}[t]{6cm}
                                                            File to visualize and compare framework results
                                                            \end{minipage}. 
    .4 \textcolor{cyan}{libs}.
    .5 \textcolor{cyan}{altera}.
    .6 \textcolor{violet}{altera\_mf{.}v} \dotfill \:\:\begin{minipage}[t]{6cm}
                                                            Library file containing help-function for memory operations, used in simulation
                                                            \end{minipage}. 
    .3 \textcolor{cyan}{methodology} \dotfill \:\:\begin{minipage}[t]{6cm}
                                                            Scripts and utilities for toolchain
                                                            \end{minipage}. 
    .2 \textcolor{teal}{CreateNewProject{.}sh} \dotfill \:\:\begin{minipage}[t]{6cm}
                                                            Script for creating new project
                                                            \end{minipage}. 
    }
\end{minipage}
\caption{Directory and file-tree of the framework}
\label{fig:frameworkdirtree}
\end{figure}

\section{HLS-script}
In order to automatically run HLS, simulation, synthesis and power-estimation, a script is created. Since LegUp is running on a VirtualBox image and not on the same servers where the other tools are located, some files and commands needs to be transferred between different locations. In the project \cite{holm2015pro} a possible solution using SSH and SCP were proposed. The script will build on this method, but first some additional preparations needs to be made. Since the VirtualBox guest is running on a local computer, a port forwarding rule has to be added to allow connections to port 22 of the guest from the Nordic Linux servers. The connections has to go through the host computer, as the VirtualBox guest does not have any direct connection to the network. The setting can either be set using the GUI of VirtualBox, or by running the following command from a command line:
\begin{verbatim}
  VBoxManage modifyvm myserver --natpf1 "ssh,tcp,,3022,,22"  
\end{verbatim}
Here "myserver" is the name of the VirtualBox VM and should be replaced with whatever you called the VM when you created it in VirtualBox. When this setting is set, it is possible to establish a connection over SSH from the Linux server directly to LegUp by using the port 3022 and the local IP-address of the computer running VirtualBox.

The standard SSH and SCP packages on Linux systems does not support passing the password as an argument to the command. To avoid the need to enter username and password each time a file is transferred using SCP or a command is executed using SSH, it is necessary to setup key-based authentication. This can be done manually, but the framework-script can also do this setup automatically if you pass the flag \textit{-s}. What the script does, is to generate a RSA key-file for the current user with \textit{ssh-keygen} and copy the generated file to LegUp using \textit{ssh-copy-id}. The password is passed to \textit{ssh-copy-id} using \textit{spawn}, \textit{expect} and \textit{send} commands.


The script is written for the bash-shell. The script has 7 main tasks; 
\begin{itemize}
    \item Generate constraint and Makefiles
    \item Run HLS-tool to generate Verilog design file
    \item Run simulation
    \item Run synthesis
    \item Run layout
    \item Run power estimation
    \item Generate readable result reports
\end{itemize}

\begin{verbatim}
  IF setup flag set:
    Perform SSH preshared-key-login setup
  END IF
  
  Transfer design file(s) to LegUp over SCP
  Transfer Makefile to LegUp over SCP
  
  WHILE loop over each constraint/architectural variation:
    Generate new HLS constraint file
    Transfer HLS constraint file to LegUp over SCP
    Connect to LegUp and run HLS Makefile
    Transfer generated Verilog
\end{verbatim}

The script is listen in \cref{lst:hlsscript} in \cref{sec:hlsscript}

\subsection{Constraint generating}
In order to run HLS with a variety of different constraints and settings, one constraint and one Makefile need to be generated for each run. To automate this process, an Excel document, \textit{constraints.xlsx}, has been created. Sheet 2 of this document contains the setup of the constraints. Here the user can select which constraints should be randomized and also set if the constraint should have a specific value. Some values are required, and must have the value specified.