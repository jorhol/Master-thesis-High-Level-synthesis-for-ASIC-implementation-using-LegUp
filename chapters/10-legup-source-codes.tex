\section{\label{lst:llvmirparserprogramcode}LLVM IR Parser program}
\lstset{language=C++,style=Cstyle}
\begin{lstlisting}
#include <iostream>
#include <fstream>
#include <string>
#include <algorithm>
#include <vector>
#include <set>
using namespace std;

int main(int argc, char *argv[]) {

    // Check for correct amount of arguments
    if (argc < 3) {
        cout << "Missing output file argument\n";
        if (argc < 2) {
            cout << "Missing input file argument\n";
        }
        cout << "Arguments should be at the form: inputfile outputfile\n";
        return 1;
    }

    vector<string> sources;
    vector<string> targets;
    vector<int> labels;

    ifstream inFile(argv[1]);
    ofstream outFile(argv[2]);

    if (inFile.is_open()) {
        cout << "inFile opened successfully\n";
        string line;
        string searchStringStores = "  store";
        string searchStringMain = "define";
        string labelString = "; <label>:";
        string whitespace = " ";
        int inMain = false;
        int currLabel = 0;
        bool isTarget = false;
        // Read file line by line
        while (getline(inFile, line)) {
            // Only consider lines staring with "  store"
            if (line.compare(0, searchStringMain.length(), searchStringMain) ==
                    0 &&
                !inMain) {

                size_t found = 0;
                do {
                    found = line.find(whitespace, found + 1);
                } while (line.compare(found + 1, 1, "@") != 0);
                if (line.compare(found + 1, 6, "@main(") == 0) {
                    inMain = true;
                    currLabel = 0;
                }
            } else if (line.compare(0, labelString.length(), labelString) ==
                       0) {
                currLabel = atoi(line.substr(10, 3).c_str());
            } else if (line.compare(0, searchStringStores.length(),
                                    searchStringStores) == 0 &&
                       inMain) {

                // Remove commas from line
                line.erase(std::remove(line.begin(), line.end(), ','),
                           line.end());

                // Remove leading and trailing whitespaces
                line.erase(
                    line.begin(),
                    std::find_if(line.begin(), line.end(),
                                 bind1st(std::not_equal_to<char>(), ' ')));

                // Split line at whitespace

                size_t found = line.find(whitespace);
                while (found != string::npos) {
                    size_t foundNext = line.find(whitespace, found + 1);

                    // Only store words staring with a % sign
                    if (line.compare(found + 1, 1, "%") == 0) {
                        string substring =
                            line.substr(found + 2, foundNext - found - 2);
                        if (isTarget == true) {
                            targets.push_back(substring);
                            labels.push_back(currLabel);
                            isTarget = false;
                        } else {
                            sources.push_back(substring);
                            isTarget = true;
                        }
                    }
                    found = foundNext;
                }
                if (sources.size() > targets.size()) {
                    sources.pop_back();
                    isTarget = false;
                }
                if (line.compare(0, 1, "}") == 0) {
                    inMain = false;
                }
            }
        }
        inFile.close();
    }

    else
        cout << "Unable to open input file\n";

    if (outFile.is_open()) {
        cout << "outFile opened successfully\n";
        set<string> done;

        // Iterate through all found stores and check for assignment connections
        for (int i = 0; i < targets.size(); ++i) {
            for (int j = 0; j < targets.size(); ++j) {
                if (targets[i] == targets[j] && i != j &&
                    done.find(sources[i]) == done.end() &&
                    sources[i].find("__out_") == 0) {
                    done.insert(sources[j]);
                    string sigName = sources[i];
                    // Only print parameters defined as outputs
                    if (sigName.find("__out_") == 0) {
                        sigName = sigName.substr(6, std::string::npos);
                        outFile << sigName << " " << sources[j] << " "
                                << labels[j] << " " << labels[i] << " "
                                << targets[i] << "\n";
                    }
                }
            }
        }
        outFile.close();
    }

    else
        cout << "Unable to open output file\n";

    return 0;
}

\end{lstlisting}
\clearpage
\section{\label{sec:validsinglssourcecode}Generating valid signals}
\lstset{language=C++,style=Cstyle}
\begin{lstlisting}
// Add each driving signal from source as a driver of the target
// signal.
// Also generate conditions for valid signals and drive these.
for (uint j = 1; j < sourceSig->getNumDrivers(); j += 2) {
    if (sourceSig->getDriver(j)->getName().compare(
            targetSig->getName()) != 0) {
        targetSig->addCondition(sourceSig->getCondition(j),
                                sourceSig->getDriver(j));
        if (j + 1 < sourceSig->getNumDrivers()) {
            targetSig->addCondition(sourceSig->getCondition(j + 1),
                                    sourceSig->getDriver(j + 1));
        }
        if (sourceSig->getCondition(j)->isOp()) {
            validSig->addCondition(sourceSig->getCondition(j), ONE);
            if (j + 1 < sourceSig->getNumDrivers()) {
                validSig->addCondition(
                    sourceSig->getCondition(j + 1), ONE);
            }
            if (sourceSig->getNumDrivers() - 1 < 2) {
                // Adds deassertion of validSig if only single
                // conditions are present.
                validSig->addCondition(
                    rtl->addOp(RTLOp::Not)
                        ->setOperands(sourceSig->getCondition(j)),
                    ZERO);
            } else if (sourceSig->getNumDrivers() - 1 < 3 ||
                       j == 1) {
                notValid->setOperands(
                    rtl->addOp(RTLOp::Not)
                        ->setOperands(sourceSig->getCondition(j)),
                    rtl->addOp(RTLOp::Not)->setOperands(
                        sourceSig->getCondition(j + 1)));
            } else if (sourceSig->getNumDrivers() - j > 1) {
                RTLSignal *notValid1 =
                    rtl->addOp(RTLOp::And)->setOperands(
                        rtl->addOp(RTLOp::Not)->setOperands(
                            sourceSig->getCondition(j)),
                        rtl->addOp(RTLOp::Not)->setOperands(
                            sourceSig->getCondition(j + 1)));
                RTLSignal *notValid2 =
                    rtl->addOp(RTLOp::And)
                        ->setOperands(notValid->getOperand(0),
                                      notValid->getOperand(1));
                notValid->setOperands(notValid1, notValid2);
            } else {
                RTLSignal *notValid1 =
                    rtl->addOp(RTLOp::And)
                        ->setOperands(notValid->getOperand(0),
                                      notValid->getOperand(1));
                notValid->setOperands(
                    notValid1, rtl->addOp(RTLOp::Not)->setOperands(
                                   sourceSig->getCondition(j)));
            }
        }
    }
}
// Adds deassertion of validSig if multiple conditions are present.
if (notValid->getNumOperands() > 1) {
    validSig->addCondition(notValid, ZERO);
}
\end{lstlisting}
\clearpage
\section{\label{sec:iterationfinishsourcecode}Adding iterationFinish flag}
\begin{lstlisting}
RTLSignal *interationFinish = rtl->addOutReg("interationFinish");

connectSignalToDriverInState(interationFinish, ONE, (--fsm->end())->getPrevNode());
interationFinish->addCondition(rtl->addOp(RTLOp::Not)->setOperands(interationFinish->getCondition(0)), ZERO);
\end{lstlisting}

\section{\label{sec:tbgenerationsourcecode}Testbench generator source code}
\begin{lstlisting}
RTLModule *t = m->addModule("main", "main_inst");
if (LEGUP_CONFIG->getParameterInt("ASIC_IMPLEMENTATION")) {
  RTLModule *rtl = alloc->getModuleForFunction(alloc->getModule()->getFunction("main"));
  if (rtl->getName().compare("main") == 0) {
    for (RTLModule::const_signal_iterator i = rtl->port_begin(), e = rtl->port_end(); i != e; ++i) {
      const RTLSignal *s = *i;
      RTLSignal *d;
      string type = s->getType();
      if (!type.empty()) {
        if (type.compare(0, 6, "output") == 0) {
          d = m->addWire(s->getName(), s->getWidth());
          t->addOut(s->getName(), s->getWidth())->connect(d);
        } else {
          d = m->addReg(s->getName(), s->getWidth());
          t->addIn(s->getName(), s->getWidth())->connect(d);
        }
      }
    }
  }
}
\end{lstlisting}
\clearpage
\section{\label{sec:hlsscriptsourcecode}Script for running framework}
\lstset{language=[gnu] make, style=Cstyle, morestring=[s]""}
\begin{lstlisting}[caption={HLS Script source code},label=lst:hlsscriptsourcecode]
#!/bin/bash
rm -f HLSscript.log
LOG_FILE=HLSscript.log
exec 3>&1 1>>${LOG_FILE} 2>&1 # Print log to file, print specified echos to terminal

DESIGNNAME=designname
REMOTEIP=192.168.12.33 # IP of the computer running the LegUp VirtualBox guest
REMOTEPORT=3022 # Port that is forwarded to port 22 on VirtualBox guest
REMOTEDIR=/home/legup/legup-4.0/examples
LEGUPUSER=legup # Username of LegUp image
LEGUPPASS=letmein # Password of LegUp image
BASE_DIR=basedir
LOCALDIR=$BASE_DIR/$DESIGNNAME/ip/$DESIGNNAME #Location of source files on Linux server.

export DESIGN_NAME=$DESIGNNAME
export FILE_LIST=$DESIGNNAME
export BASE_DIR=$BASE_DIR
export VC_WORKSPACE=$BASE_DIR/$DESIGNNAME

module load icc # Load IC compiler module
module load primetime # Load PrimeTime module

SSHCOMMANDS2="mkdir $REMOTEDIR/$DESIGNNAME; cd $REMOTEDIR/$DESIGNNAME/; libreoffice --headless --convert-to csv constraints.xlsx --outdir .; exit" # ssh commands for converting excel file to csv

if [$1 = "-s"]; then
	echo Setup started
	ssh-keygen -f id_rsa -t rsa -N ''
	spawn ssh-copy-id "$LEGUPUSER@$REMOTEIP -p $REMOTEPORT"
	expect "password:"
	send "$LEGUPPASS\n"
	expect eof
	echo Setup finished
fi

mkdir -p $LOCALDIR/hls/
ssh $LEGUPUSER@$REMOTEIP -p $REMOTEPORT "mkdir -p $REMOTEDIR/$DESIGNNAME"
scp -P $REMOTEPORT $LOCALDIR/$DESIGNNAME.c $LEGUPUSER@$REMOTEIP:$REMOTEDIR/$DESIGNNAME #Copy design file to LegUp image
scp -P $REMOTEPORT $LOCALDIR/sim/tb/test_$DESIGNNAME\_testcases.v $LEGUPUSER@$REMOTEIP:$REMOTEDIR/$DESIGNNAME #Copy testcases file to LegUp

scp -P $REMOTEPORT $LOCALDIR/hls/constraints.xlsx $LEGUPUSER@$REMOTEIP:$REMOTEDIR/$DESIGNNAME/ #Copy design constraint definitions to LegUp image
ssh $LEGUPUSER@$REMOTEIP -p $REMOTEPORT $SSHCOMMANDS2 #Run commands and script for generating constraint and Makefiles
scp -P $REMOTEPORT $LEGUPUSER@$REMOTEIP:$REMOTEDIR/$DESIGNNAME/constraints.csv $LOCALDIR/hls/ #Copy CSV file from LegUp image
sed 's/\''//g' -i $LOCALDIR/hls/constraints.csv # Remove excess quotes
rm -r $LOCALDIR/hls/makefiles $LOCALDIR/hls/constraintfiles
mkdir $LOCALDIR/hls/makefiles $LOCALDIR/hls/constraintfiles
mkdir $LOCALDIR/reports
rm $LOCALDIR/reports/*.rpt
cd $LOCALDIR/hls/

$LOCALDIR/hls/constraintsGenerator.run $LOCALDIR/hls/constraints.csv .. $DESIGNNAME
NUMRUNS=$?
echo "Generated $NUMRUNS constraint and Makefiles" | tee /dev/fd/3
COUNTER=0
while [ $COUNTER -lt $NUMRUNS ]; do
	echo "Framework loop #$COUNTER" 1>&3
	rm $LOCALDIR/rtl/{*.tcl,*.v,*.mif}
	SSHCOMMANDS="export PATH=/home/legup/clang+llvm-3.5.0-x86_64-linux-gnu/bin:$PATH; cd $REMOTEDIR/$DESIGNNAME/; make clean; make; exit" # Commands to run on SSH session. Need to add clang to PATH as this is not present in SSH session.
	scp -P $REMOTEPORT $LOCALDIR/hls/constraintfiles/config$COUNTER.tcl $LEGUPUSER@$REMOTEIP:$REMOTEDIR/$DESIGNNAME/ #Copy design constraint file to LegUp image
	scp -P $REMOTEPORT $LOCALDIR/hls/makefiles/Makefile$COUNTER $LEGUPUSER@$REMOTEIP:$REMOTEDIR/$DESIGNNAME/Makefile #Copy design Makefile to LegUp image
	echo "Running HLS" 1>&3
	ssh $LEGUPUSER@$REMOTEIP -p $REMOTEPORT $SSHCOMMANDS #Run LegUp
	
	scp -P $REMOTEPORT $LEGUPUSER@$REMOTEIP:$REMOTEDIR/$DESIGNNAME/$DESIGNNAME.v $LOCALDIR/rtl/ #Copy Verilog file from LegUp image
	scp -P $REMOTEPORT $LEGUPUSER@$REMOTEIP:$REMOTEDIR/$DESIGNNAME/test_main.v $LOCALDIR/sim/tb/test_$DESIGNNAME.v #Copy Verilog testbench file from LegUp image
	
	MEM_CRTL_EXIST=$(grep -c "module memory_controller" $LOCALDIR/rtl/$DESIGNNAME.v)
	if [ $MEM_CRTL_EXIST -gt 0 ]; then
		echo "memory_controller module exist in design. Please check your design" 1>&3
	fi
	
	find $LOCALDIR/rtl/$DESIGNNAME.v -type f -exec sed -i "s/module main/module $DESIGNNAME/g" {} \; #Replace top modulename main with designname
	
	find $LOCALDIR/sim/tb/test_$DESIGNNAME.v -type f -exec sed -i "s/module main_tb/module test_$DESIGNNAME/g" {} \; #Replace tb declaration with correct designname
	find $LOCALDIR/sim/tb/test_$DESIGNNAME.v -type f -exec sed -i "s/main main_inst/$DESIGNNAME u_$DESIGNNAME/g" {} \; #Replace top module instantiation in tb with correct designname
	
	echo "Running simulation" 1>&3
	#Run simulation
	(cd $LOCALDIR/sim/run/ && (RUN_ALL --clean) && (vcd2saif -input $LOCALDIR/sim/run/$DESIGNNAME.vcd -output $LOCALDIR/sim/run/$DESIGNNAME.saif)) 
	
	echo "Running synthesis" 1>&3
	#Run synthesis
	(cd $LOCALDIR/syn/ && (make clean) && (make compile)) #Run synthesis clean removes old data
	
	echo "Running layout" 1>&3
	#Run layout
	(cd $LOCALDIR/lay/ && (make clean) && (make outputs_cts))
	
	echo "Running power analysis" 1>&3
	#Run power estimation
	(cd $LOCALDIR/pow/ && (make clean) && (make power_analysis))
	
	#Store synthesis results to common file
	
	echo "Gathering layout results" 1>&3
	var1=$(grep "Combinational Area:" $LOCALDIR/lay/reports/clock_opt_cts_icc.qor)
	var1=${var1//  Combinational Area:/}
	var1=${var1// /}
	var1=${var1//./,}
	echo $var1 >> $LOCALDIR/reports/noncombinational_area.rpt
	var2=$(grep "Noncombinational Area:" $LOCALDIR/lay/reports/clock_opt_cts_icc.qor)
	var2=${var2//  Noncombinational Area:/}
	var2=${var2// /}
	var2=${var2//./,}
	echo $var2 >> $LOCALDIR/reports/combinational_area.rpt
	var3=$(grep "Design Area:" $LOCALDIR/lay/reports/clock_opt_cts_icc.qor)
	var3=${var3//  Design Area: /}
	var3=${var3// /}
	var3=${var3//./,}
	echo $var3 >> $LOCALDIR/reports/design_area.rpt
	var4=$(grep "Total number of registers" $LOCALDIR/syn/reports/$DESIGNNAME.mapped.clock_gating.rpt)
	var4=${var4//          |    Total number of registers          |/}
	var4=${var4// /}
	var4=${var4//|/}
	echo $var4 >> $LOCALDIR/reports/register_count.rpt
	
	echo "Gathering power analysis results" 1>&3
	
	COUNT=0
	while [ $COUNT -lt 4 ]; do
		swpow=$(grep 'Net Switching Power' $LOCALDIR/pow/reports/power_analysis_$DESIGNNAME\_ctrl$COUNT/power_summary.rpt)
		swpow=${swpow//([^)]*)/}
		swpow=${swpow//  Net Switching Power  = /}
		echo -n "$swpow\t">>$LOCALDIR/reports/net_switching_power.rpt
		intpow=$(grep 'Cell Internal Power' $LOCALDIR/pow/reports/power_analysis_$DESIGNNAME\_ctrl$COUNT/power_summary.rpt)
		intpow=${intpow//([^)]*)/}
		intpow=${intpow//  Cell Internal Power  = /}
		echo -n "$intpow\t">>$LOCALDIR/reports/cell_internal_power.rpt
		leakpow=$(grep 'Cell Leakage Power' $LOCALDIR/pow/reports/power_analysis_$DESIGNNAME\_ctrl$COUNT/power_summary.rpt)
		leakpow=${leakpow//([^)]*)/}
		leakpow=${leakpow//  Cell Leakage Power   = /}
		echo -n "$leakpow\t">>$LOCALDIR/reports/cell_leakage_power.rpt
		totpow=$(grep 'Total Power' $LOCALDIR/pow/reports/power_analysis_$DESIGNNAME\_ctrl$COUNT/power_summary.rpt)
		totpow=${totpow//([^)]*)/}
		totpow=${totpow//Total Power            = /}
		echo -n "$totpow\t">>$LOCALDIR/reports/total_power.rpt
		let COUNT=COUNT+1 
	done
	
	swpow=$(grep 'Net Switching Power' $LOCALDIR/pow/reports/power_analysis_$DESIGNNAME\_inactive/power_summary.rpt)
	swpow=${swpow//([^)]*)/}
	swpow=${swpow//  Net Switching Power  = /}
	echo $swpow>>$LOCALDIR/reports/net_switching_power.rpt
	intpow=$(grep 'Cell Internal Power' $LOCALDIR/pow/reports/power_analysis_$DESIGNNAME\_inactive/power_summary.rpt)
	intpow=${intpow//([^)]*)/}
	intpow=${intpow//  Cell Internal Power  = /}
	echo $intpow>>$LOCALDIR/reports/cell_internal_power.rpt
	leakpow=$(grep 'Cell Leakage Power' $LOCALDIR/pow/reports/power_analysis_$DESIGNNAME\_inactive/power_summary.rpt)
	leakpow=${leakpow//([^)]*)/}
	leakpow=${leakpow//  Cell Leakage Power   = /}
	echo $leakpow>>$LOCALDIR/reports/cell_leakage_power.rpt
	totpow=$(grep 'Total Power' $LOCALDIR/pow/reports/power_analysis_$DESIGNNAME\_inactive/power_summary.rpt)
	totpow=${totpow//([^)]*)/}
	totpow=${totpow//Total Power            = /}
	echo $totpow>>$LOCALDIR/reports/total_power.rpt
		
	echo "Register count\tCombinational Area\tNon-combinational Area\tDesign Area\tSwitching Power\tInternal Power\tLeakage Power\tTotal Power" > all_results.rpt
	paste register_count.rpt combinational_area.rpt noncombinational_area.rpt design_area.rpt net_switching_power.rpt cell_internal_power.rpt cell_leakage_power.rpt total_power.rpt >> all_results.rpt
	
	#Store results in dedicated folder
	rm -f $LOCALDIR/sim/run/$DESIGNNAME.vcd #VCD file can get large. Remove before storing framework run data.
	mkdir -p $LOCALDIR/hls/rtl$COUNTER/
	cp $LOCALDIR/hls/constraintfiles/config$COUNTER.tcl $LOCALDIR/rtl/ #Copy design constraint file to current rtl folder
	cp $LOCALDIR/hls/makefiles/Makefile$COUNTER $LOCALDIR/rtl/Makefile
	cp -r $LOCALDIR/{rtl/,sim/,syn/,lay/,pow/,score/} $LOCALDIR/hls/rtl$COUNTER/
	
	let COUNTER=COUNTER+1 
done
echo HLS finished
exit $?
\end{lstlisting}
\clearpage

\section{\label{sec:constraintgeneratorsourcecode}Constraint-generator program}
\lstset{language=C++, style=Cstyle}
\begin{lstlisting}[caption={Constraint-generation program source code},label=lst:constraintGenerating]
#include <stdio.h>
#include <stdlib.h>
#include <iostream>
#include <fstream>
#include <math.h>
#include <sstream>
#include <map>
#include <vector>

using namespace std;

map<string, string> requiredConstraints;
vector<string> randomConstraints;
map<string, string> staticConstraints;
map<string, string> makefileConstraints;
map<string, string> nonParameterConstraints;

int main(int argc, char *argv[]) {

    // Check for correct amount of arguments
    if (argc < 4) {
        cout << "Missing design-name argument\n";
        if (argc < 3) {
            cout << "Missing Makefile LEVEL argument\n";
            if (argc < 1) {
                cout << "Missing constraints csv-fileName argument\n";
            }
        }
        cout << "Arguments should be at the form: csv-fileName LEVEL "
                "design-name\n";
        return 0;
    }

    // Read constraints from .csv file
    ifstream csvFile;
    csvFile.open(argv[1]);

    while (csvFile) {
        string s;
        if (!getline(csvFile, s))
            break;

        istringstream ss(s);
        vector<string> record;

        while (ss) {
            string s;
            if (!getline(ss, s, ','))
                break;
            record.push_back(s);
        }
        bool required = false;
        bool isParameter = false;
        bool isMakefile = false;
        string value = record.back();
        record.pop_back();
				if (value == "discard") {
            continue;
        }
        if (value == "makefile") {
            isMakefile = true;
            value = record.back();
            record.pop_back();
        }
        if (value == "parameter") {
            isParameter = true;
            value = record.back();
            record.pop_back();
        }
        if (value == "required") {
            required = true;
            value = record.back();
            record.pop_back();
        }
        string parameter = record.back();
        record.pop_back();

        if (value == "random") {
            randomConstraints.push_back(parameter);
        } else if (required == true) {
            requiredConstraints.insert(
                std::pair<string, string>(parameter, value));
        } else {
            if (isMakefile == true) {
                makefileConstraints.insert(
                    std::pair<string, string>(parameter, value));
            } else if (isParameter == true) {
                staticConstraints.insert(
                    std::pair<string, string>(parameter, value));
            } else {
                nonParameterConstraints.insert(
                    std::pair<string, string>(parameter, value));
            }
        }
    }

    csvFile.close();

    // Generate constraint-files

    ofstream constraintFile;
    ofstream makeFile;
    char buffer[100];
    int n = sprintf(buffer, "%d", (int)pow(2, randomConstraints.size()));
    for (int count = 0; count < pow(2, randomConstraints.size()); count++) {
        sprintf(buffer, "%d", count); //sprintf(buffer, "%.*d", n, count);
        string cFileName = "config" + string(buffer) + ".tcl";
        string fileLocation = "./constraintfiles/" + cFileName;
        constraintFile.open(fileLocation.c_str());
        constraintFile << "source " << argv[2] << "/legup.tcl\n\n"
                       << "###################################################"
                          "#################\n"
                       << "## Required Constraints:\n";
        for (std::map<string, string>::iterator it =
                 requiredConstraints.begin();
             it != requiredConstraints.end(); ++it) {
            constraintFile << "set_parameter " << it->first << " " << it->second
                           << "\n";
        }
        constraintFile << "\n\n###############################################"
                          "#####################\n"
                       << "## Random Constraints:\n";

        for (int offset = randomConstraints.size() - 1; offset >= 0; offset--) {
            constraintFile << "set_parameter " << randomConstraints[offset]
                           << " " << ((count & (1 << offset)) >> offset)
                           << "\n";
        }
        constraintFile << "\n\n###############################################"
                          "#####################\n"
                       << "## Static Parameter Constraints:\n";
        for (std::map<string, string>::iterator it = staticConstraints.begin();
             it != staticConstraints.end(); ++it) {
            constraintFile << "set_parameter " << it->first << " " << it->second
                           << "\n";
        }
        constraintFile << "\n\n###############################################"
                          "#####################\n"
                       << "## Static Non-parameter Constraints:\n";
        for (std::map<string, string>::iterator it =
                 nonParameterConstraints.begin();
             it != nonParameterConstraints.end(); ++it) {
            constraintFile << it->first << " " << it->second << "\n";
        }
        constraintFile.close();

        // Generate Makefile for each constraint
        string mFileName = "./makefiles/Makefile" + string(buffer);
        makeFile.open(mFileName.c_str());

        makeFile << "#########################################################"
                    "###########\n"
                 << "## Generated makefile:\n"
                 << "NAME=" << argv[3] << "\n"
                 << "LEVEL = " << argv[2] << "\n";
        for (std::map<string, string>::iterator it =
                 makefileConstraints.begin();
             it != makefileConstraints.end(); ++it) {
            makeFile << it->first << "=" << it->second << "\n";
        }
        makeFile << "LOCAL_CONFIG = -legup-config=" << cFileName << "\n"
                 << "include $(LEVEL)/Makefile.common\n";
        makeFile.close();
    }
    return (int) pow(2, randomConstraints.size());
}
\end{lstlisting}

