\chapter{Appendix}
\section{\label{sec:sourcecode}Reference design source codes}

\subsection{FIR-filter - Verilog}
\lstset{language=Verilog, style=Verilogstyle}
\begin{lstlisting}[caption={FIR-filter implemented in Verilog},label=lst:firfilterverilog]
module VerilogFIR (
		ck,
		arst,
		dataIn,
		dataOut
	);
	input ck, arst;
	input signed [WIDTH-1:0] dataIn;
	output reg signed [2*WIDTH-1+$clog2(DEPTH):0] dataOut;
	
	parameter	WIDTH = 32;
	parameter	DEPTH = 4;
	parameter logic signed [WIDTH-1:0] COEFF [DEPTH-1:0] = {1,2,3,4};
	
	reg signed [WIDTH-1:0] sr [DEPTH-2:0];
	reg signed [2*WIDTH-1:0] products [DEPTH-1:0];
	reg signed [2*WIDTH-1+$clog2(DEPTH):0] sum;
	
	always @(posedge ck or arst) begin
		if(arst == 1'b1) begin
			sum = 0;
		end 
		else begin
			sr[0] <= dataIn;
			sr[DEPTH-2:1] <= sr[DEPTH-3:0];
		end
	end
	
	always @* begin
		for (int i = 1; i < DEPTH; i++)
			products[i] = sr[i-1] * COEFF[i];
	end
	
	always @* begin
		sum = products[0];
		for (int i = 1; i < DEPTH; i++)
			sum = sum + products[i];
	end
	
	assign dataOut = sum;
endmodule
\end{lstlisting}
\clearpage
\subsection{FIR-filter - C}
\lstset{language=C,style=Cstyle}
\begin{lstlisting}[caption={FIR-filter implemented in C},label=lst:firfilterc]
#define WIDTH 32  //implicit from the use of ints as input,
#define DEPTH 4

int main(int inData, char **test){
    int sr[DEPTH-1];
	int products[DEPTH];
    int coeff[DEPTH] = {1,2,3,4};

    int sum;
	int i, j;
	products[0] = dataIn * coeff[0];
	sum = products[0];
    for (i = 1; i <= DEPTH; i++){
		for(j=DEPTH-2; j >= 1; j-=1 ){
			sr[j] = sr[j-1];
		}
		sr[0] = inData;
		for (j = 1; j < DEPTH; j++){
			products[i] += sr[j-1] * coeff[j];
		}
        sum += products[i];
    }
    return sum;
}
\end{lstlisting}
\clearpage
\subsection{SAP-1 architecture - Verilog}
\lstset{language=Verilog,style=Verilogstyle}
\begin{lstlisting}[caption={SAP-1 architecture implemented in C},label=lst:sap1archverilog]
module Sap1ProgramCounter (ck, arst, cp, ep, wbus);
	parameter 					BUS_WIDTH 	= 8;
	input 						ck, arst, cp, ep;
	output	[BUS_WIDTH/2-1:0]	wbus;
	reg		[BUS_WIDTH/2-1:0]	pc;
	always @ (posedge ck)
	begin
		if(arst)		pc = 4'b0000;
		else if (cp)	pc = pc + 1;
	end
	assign wbus = ep ? pc : 4'bzzzz;
endmodule

module Sap1MemoryAddressRegister(ck, lm_n, ramAddress, wbus);
	parameter 					BUS_WIDTH 	= 8;
	parameter 					ADDR_WIDTH 	= 4;
	input 						ck, lm_n;
	input	[BUS_WIDTH/2-1:0]	wbus;
	output	[ADDR_WIDTH-1:0]	ramAddress;
	reg		[ADDR_WIDTH-1:0]	ramAddress;
	always @ (posedge ck)
	begin 
		if(!lm_n)	ramAddress = wbus;
	end
endmodule 

module Sap1Ram(ck, ce_n, addr, wbus);

	parameter 					BUS_WIDTH	= 8;
	parameter 					ADDR_WIDTH	= 4;
	input 						ck, ce_n;
	input	[ADDR_WIDTH-1:0]	addr;
	inout	[BUS_WIDTH-1:0] 	wbus;
	reg		[BUS_WIDTH-1:0] 	dataout;
	reg		[BUS_WIDTH-1:0] 	rammem[0:2**ADDR_WIDTH];
	always @ (ce_n)      //asynchronous RAM
	begin 
		dataout <= !ce_n ? rammem[addr] : 8'bz;
	end
	assign wbus = dataout;

endmodule

module Sap1InstructionRegister(ck, arst, li_n, ei_n, opcode, wbus);
	parameter 					BUS_WIDTH = 8;
	input 						ck, li_n, ei_n, arst;
	inout	[BUS_WIDTH-1:0] 	wbus;
	output	[BUS_WIDTH/2-1:0]	opcode;
	reg		[BUS_WIDTH-1:0]		wbusOut;
	reg		[BUS_WIDTH-1:0]		instructionRegister;
	
	always @(posedge ck)
	begin
		if (arst)
		begin 
			instructionRegister <= 8'b0;
		end
		else if(!li_n) 
		begin
			instructionRegister <= wbus;
		end
	end
	assign wbus = !ei_n?{4'b0,instructionRegister[BUS_WIDTH/2-1:0]}:8'bz;
	assign opcode = instructionRegister[BUS_WIDTH-1:BUS_WIDTH/2-1];
endmodule

module Sap1AccumulatorA (ck, la_n, ea, registerA, wbus);
	parameter 				BUS_WIDTH = 8;
	input 					ck, la_n, ea;
	inout	[BUS_WIDTH-1:0]	wbus;
	output 					registerA;
	reg		[BUS_WIDTH-1:0]	registerA;
	always @(posedge ck)
	begin 
		if(!la_n) 
		begin
			registerA <= wbus;
		end
	end
	assign wbus = ea?registerA:8'bz;
endmodule

module Sap1AddSub(su, eu, registerA, registerB, wbus);
	parameter 				BUS_WIDTH = 8;
	input 					su, eu;
	input	[BUS_WIDTH-1:0] registerA, registerB;
	output	[BUS_WIDTH-1:0] wbus;
	assign wbus = eu?(su?registerA-registerB:registerA+registerB):8'bz;
endmodule

module Sap1BRegister(ck, lb_n, registerB, wbus);
	parameter 				BUS_WIDTH = 8;
	input 					ck, lb_n;
	input	[BUS_WIDTH-1:0] wbus;
	output	[BUS_WIDTH-1:0] registerB;
	reg		[BUS_WIDTH-1:0] registerB;
	always @ (posedge ck)
	begin 
		if(!lb_n) registerB = wbus;
	end
endmodule

module Sap1OutputRegister(ck, lo_n, out, wbus);
	parameter 				BUS_WIDTH = 8;
	input 					ck, lo_n;
	input	[BUS_WIDTH-1:0] wbus;
	output	[BUS_WIDTH-1:0] out;
	reg		[BUS_WIDTH-1:0] out;
	always @ (posedge ck)
	begin 
		if(!lo_n) out = wbus;
	end
endmodule

module Sap1ControllerSequencer(ck,  arst, controlBus, opcode);
	parameter OPCODE_WIDTH = 4;
	parameter CONTROL_BUS_WIDTH = 12;
	parameter T1 = 6'b000001,
			  T2 = 6'b000010,
			  T3 = 6'b000100,
			  T4 = 6'b001000,
			  T5 = 6'b010000,
			  T6 = 6'b100000;
	input							ck,arst;
	input	[OPCODE_WIDTH-1:0]		opcode;
	output	[CONTROL_BUS_WIDTH-1:0]	controlBus;
	reg		[CONTROL_BUS_WIDTH-1:0] controlBus;
	// ring counter 
	reg		[5:0]  					ringCount;
	wire	[5:0] 					state;
	always @ (posedge ck)
	begin
		if (arst)
			ringCount <= T1;
		else 
		begin case(ringCount)
			T1: ringCount <= T2;
			T2: ringCount <= T3;
			T3: ringCount <= T4;
			T4: ringCount <= T5;
			T5: ringCount <= T6; 
			T6: ringCount <= T1;
			endcase
		end
	end
	assign state = ringCount;
	always @(posedge ck)
	begin
		case ({state, opcode})
		{T1, 4'hx}: controlBus = 12'b010111100011;
		{T2, 4'hx}: controlBus = 12'b101111100011;
		{T3, 4'hx}: controlBus = 12'b001001100011;
		//LDA operation
		{T4, 4'h0}: controlBus = 12'b000110100011;
		{T5, 4'h0}: controlBus = 12'b001011000011;
		{T6, 4'h0}: controlBus = 12'b001111100011;
		//ADD
		{T4, 4'h1}: controlBus = 12'b000110100011;
		{T5, 4'h1}: controlBus = 12'b001011100001;
		{T6, 4'h1}: controlBus = 12'b001111000111;
		//SUB
		{T4, 4'h2}: controlBus = 12'b000110100011;
		{T5, 4'h2}: controlBus = 12'b001011100001;
		{T6, 4'h2}: controlBus = 12'b001111001111;
		//OUT
		{T4, 4'hE}: controlBus = 12'b001111110010;
		{T5, 4'hE}: controlBus = 12'b001111100011;
		{T6, 4'hE}: controlBus = 12'b001111100011;
		//HLT
		{T4, 4'hF}: controlBus = 12'b001111100011;
		{T5, 4'hF}: controlBus = 12'b001111100011;
		{T6, 4'hF}: controlBus = 12'b001111100011;
		endcase
	end
endmodule

module CFIR(ck, arst, result); // Top module
	parameter BUS_WIDTH = 8;
	parameter CONTROL_BUS_WIDTH = 12;
	input ck, arst;
	output [BUS_WIDTH-1:0] result;
	wire [BUS_WIDTH-1:0] wbus;
	wire cp, ep, lm_n, ce_n, li_n, ei_n, la_n, ea, su, eu, lb_n, lo_n;

	wire [CONTROL_BUS_WIDTH-1:0] controlBus = {cp, ep, lm_n, ce_n, li_n, ei_n, la_n, ea, su, eu, lb_n, lo_n};
	wire [BUS_WIDTH/2-1:0] opcode;
	wire [BUS_WIDTH/2-1:0] ramAddress;
	wire [BUS_WIDTH-1:0]	registerA, registerB;

	// Instantiate modules:
	Sap1ProgramCounter 			u1_ProgramCounter(			.ck(ck), 
															.arst(arst),
															.cp(cp),
															.ep(ep),
															.wbus(wbus[BUS_WIDTH/2-1:0])
															);
	Sap1MemoryAddressRegister	u1_MemoryAddressRegister(	.ck(ck),
															.lm_n(lm_n),
															.ramAddress(ramAddress),
															.wbus(wbus[BUS_WIDTH/2-1:0])
															);
	Sap1Ram                     u1_Ram(                     .ck(ck),
															.ce_n(ce_n), 
															.addr(ramAddress),
															.wbus(wbus)
															);
	Sap1InstructionRegister 	u1_InstructionRegister(		.ck(ck),
															.arst(arst),
															.li_n(li_n),
															.ei_n(ei_n),
															.opcode(opcode),
															.wbus(wbus)
															);
	Sap1ControllerSequence		u1_ControllerSequence(		.ck(ck),
															.arst(arst),
															.controlBus(controlBus),
															.opcode(opcode)
															);
	Sap1AccumulatorA			u1_AccumulatorA(			.ck(ck),
															.la_n(la_n),
															.ea(ea),
															.registerA(registerA),
								                            .wbus(wbus)
															);
	Sap1AddSub					u1_AddSub(					.su(su),
															.eu(eu),
															.registerA(registerA),
															.registerB(registerB),
															.wbus(wbus)
															);
	Sap1BRegister 				u1_BRegister(				.ck(ck),
															.lb_n(lb_n),
															.registerB(registerB),
															.wbus(wbus)
															);
	Sap1OutputRegister 			u1_OutputRegister(			.ck(ck),
															.lo_n(lo_n),
															.out(result),
															.wbus(wbus)
															);

endmodule

\end{lstlisting}
\clearpage
\subsection{SAP-1 architecture - C}
\lstset{language=C,style=Cstyle}
\begin{lstlisting}[caption={SAP-1 architecture implemented in C},label=lst:sap1archc]
#include <stdbool.h>

typedef enum {T1, T2, T3, T4, T5, T6} states_t;
typedef enum {LDA = 0b0000, ADD = 0b0001, SUB = 0b0010, OUT = 0b1110, HLT = 0b1111} opcodes_t;

states_t state;
opcodes_t opcode;
bool cp,ep,lm_n,ce_n,li_n,ei_n,la_n,ea,su,eu,lb_n,lo_n; 
 
unsigned char wbus = 0x00;

int pc;
signed char accumulator_reg;
signed char registerb_reg;
unsigned char memoryAddressRegister_reg;
unsigned char ram[16];
 
int main (int arst) {
	while(1){
		//Reset to default values
		if(arst) {
			state = T1;
			{cp,ep,lm_n,ce_n,li_n,ei_n,la_n,ea,su,eu,lb_n,lo_n} = {0,0,1,1,1,1,1,0,0,0,1,1};
			wbus = 0x00;
			accumulator_reg = 0x00;
			registerb_reg = 0x00;
			memoryAddressRegister_reg = 0x00;
			ram = {0x00};
		}
		
		ControllerSequence();
		ProgramCounter();
		InstructionRegister();
		MemoryAddressRegister();
		Ram();
		BRegister();
		AccumulatorA();
		AddSub();
		return OutputRegister();
		state++;
	}
}

void AccumulatorA() {
	if (ea) {
		wbus = accumulator_reg;
	}
	else if (!la_n) {
		accumulator_reg = wbus;
	}
}

void ProgramCounter() {
	if(arst || pc == 15) {
		pc = 0;
	}
	else if(cp) {
		pc++;
	}
	if(ep) {
		wbus = pc;
	}
}

void MemoryAddressRegister() {
	if(!lm_n) {
		memoryAddressRegister_reg = wbus & 0b00001111;
	}
}

void AddSub() {
	if(eu){
		if(su){wbus = accumulator_reg + ~registerb_reg;}
		else if (!su) {wbus = accumulator_reg + registerb_reg;}
	}
}

void Ram() {
	if (!ce_n) {
		wbus = ram[memoryAddressRegister_reg];
	}
}

void ControllerSequence() {
	if(arst){
		state = T1;
	}
	else {
	switch(state) {
		case T1:
			{cp,ep,lm_n,ce_n,li_n,ei_n,la_n,ea,su,eu,lb_n,lo_n} = {0,1,0,1,1,1,1,0,0,0,1,1};
			break;
		case T2:
			{cp,ep,lm_n,ce_n,li_n,ei_n,la_n,ea,su,eu,lb_n,lo_n} = {1,0,1,1,1,1,1,0,0,0,1,1};
			break;
		case T3:
			{cp,ep,lm_n,ce_n,li_n,ei_n,la_n,ea,su,eu,lb_n,lo_n} = {0,0,1,0,0,1,1,0,0,0,1,1};
			break;
		case T4:
			if(opcode == LDA || opcode == ADD || opcode == SUB) {{cp,ep,lm_n,ce_n,li_n,ei_n,la_n,ea,su,eu,lb_n,lo_n} = {0,0,0,1,1,0,1,0,0,0,1,1};}
			else if(opcode == OUT) {{cp,ep,lm_n,ce_n,li_n,ei_n,la_n,ea,su,eu,lb_n,lo_n} = {0,0,1,1,1,1,0,0,0,0,1,0};}
			else {{cp,ep,lm_n,ce_n,li_n,ei_n,la_n,ea,su,eu,lb_n,lo_n} = {0,0,1,1,1,1,1,0,0,0,1,1};}
			break;
		case T5:
			if(opcode == LDA) {{cp,ep,lm_n,ce_n,li_n,ei_n,la_n,ea,su,eu,lb_n,lo_n} = {0,0,1,0,1,1,0,0,0,0,1,1};}
			else if(opcode == ADD || opcode == SUB) {{cp,ep,lm_n,ce_n,li_n,ei_n,la_n,ea,su,eu,lb_n,lo_n} = {0,0,1,0,1,1,1,0,0,0,0,1};}
			else {{cp,ep,lm_n,ce_n,li_n,ei_n,la_n,ea,su,eu,lb_n,lo_n} = {0,0,1,1,1,1,1,0,0,0,1,1};}
			break;
		case T6:
			if(opcode == ADD) {{cp,ep,lm_n,ce_n,li_n,ei_n,la_n,ea,su,eu,lb_n,lo_n} = {0,0,1,1,1,1,1,0,0,1,1,1};}
			else if(opcode == SUB) {{cp,ep,lm_n,ce_n,li_n,ei_n,la_n,ea,su,eu,lb_n,lo_n} = {0,0,1,1,1,1,1,0,1,1,1,1};}
			else {{cp,ep,lm_n,ce_n,li_n,ei_n,la_n,ea,su,eu,lb_n,lo_n} = {0,0,1,1,1,1,1,0,0,0,1,1};}
			break;
		}
	}
}

void BRegister() {
	if (!lb_n) {
		registerb_reg = wbus;
	}
}

void InstructionRegister() {
	if(arst) {
		opcode = 0x0;
	}
	else if (!li_n) {
		opcode = wbus >>> 4;
	}
	else if(!ei_n) {
		wbus = opcode;
	}
}

char OutputRegister() {
	if(ea) {
		return accumulator_reg;
	}
	else {
		return NULL;
	}
}
\end{lstlisting}
\clearpage
\section{SystemVerilog example: FIR-filter}
\lstset{language=SystemVerilog, style=Verilogstyle}
\begin{lstlisting}[caption={FIR-filter implemented in SystemVerilog. Example from \cite{mehler2014digital}},label=lst:firfiltersystemverilog]
module direct2 #(WIDTH = 4, DEPTH = 4)
  (
	input ck, 
	input signed [WIDTH-1:0] dataIn,
	output logic signed[2*WIDTH-1+$clog2(DEPTH):0] dataOut
	);
	
	logic signed [WIDTH-1:0] sr [DEPTH-2:0];
	logic signed [2*WIDTH-1:0] products [DEPTH-1:0];
	parameter logic signed [WIDTH-1:0] coeff [DEPTH-1:0] = {1,2,3,4};
	logic signed [2*WIDTH-1 + $clog2(DEPTH):0] sum;
	// Test
	always_ff @(posedge ck) begin
		sr[0] <= dataIn;
		sr[DEPTH-2:1] <= sr[DEPTH-3:0]:
	end
	
	always_comb begin
		products[0] = dataIn * coeff[0];
		for (int i = 1; i < DEPTH; i++)
			products[i] = sr[i-1] * coeff[i];
	end
	
	always_comb begin
		sum = products[0]
		for (int i = 1; i < DEPTH; i++)
			sum = sum + products[i];
	end
	
	always_comb dataOut = sum;
endmodule
\end{lstlisting}
\clearpage
\section{\label{sec:ramtempcode}True dual-port byte-enable RAM template}
\lstset{language=SystemVerilog,style=Verilogstyle}
\begin{lstlisting}
// Quartus II SystemVerilog Template
//
// True Dual-Port RAM with different read/write addresses and single read/write clock
// and with a control for writing single bytes into the memory word; byte enable

// Read during write produces old data on ports A and B and old data on mixed ports
// For device families that do not support this mode (e.g. Stratix V) the ram is not inferred

module byte_enabled_true_dual_port_ram
	#(
		parameter int
		BYTE_WIDTH = 8,
		ADDRESS_WIDTH = 6,
		BYTES = 4,
		DATA_WIDTH_R = BYTE_WIDTH * BYTES
)
(
	input [ADDRESS_WIDTH-1:0] addr1,
	input [ADDRESS_WIDTH-1:0] addr2,
	input [BYTES-1:0] be1,
	input [BYTES-1:0] be2,
	input [BYTE_WIDTH-1:0] data_in1, 
	input [BYTE_WIDTH-1:0] data_in2, 
	input we1, we2, clk,
	output [DATA_WIDTH_R-1:0] data_out1,
	output [DATA_WIDTH_R-1:0] data_out2);
	localparam RAM_DEPTH = 1 << ADDRESS_WIDTH;

	// model the RAM with two dimensional packed array
	logic [BYTES-1:0][BYTE_WIDTH-1:0] ram[0:RAM_DEPTH-1];

	reg [DATA_WIDTH_R-1:0] data_reg1;
	reg [DATA_WIDTH_R-1:0] data_reg2;

	// port A
	always@(posedge clk)
	begin
		if(we1) begin
		// edit this code if using other than four bytes per word
			if(be1[0]) ram[addr1][0] <= data_in1;
			if(be1[1]) ram[addr1][1] <= data_in1;
			if(be1[2]) ram[addr1][2] <= data_in1;
			if(be1[3]) ram[addr1][3] <= data_in1;
		end
	data_reg1 <= ram[addr1];
	end

	assign data_out1 = data_reg1;
   
	// port B
	always@(posedge clk)
	begin
		if(we2) begin
		// edit this code if using other than four bytes per word
			if(be2[0]) ram[addr2][0] <= data_in2;
			if(be2[1]) ram[addr2][1] <= data_in2;
			if(be2[2]) ram[addr2][2] <= data_in2;
			if(be2[3]) ram[addr2][3] <= data_in2;
		end
	data_reg2 <= ram[addr2];
	end

	assign data_out2 = data_reg2;

endmodule : byte_enabled_true_dual_port_ram
\end{lstlisting}
\lstset{language=LLVM,style=Cstyle}
\section{LLVM IR}
\lstset{language=LLVM,style=Cstyle}
\begin{lstlisting}[caption={Intermediate Representation from LLVM.}, label={lst:llvmir}]
; ModuleID = 'sra.bc'
target datalayout = "e-m:e-p:32:32-f64:32:64-f80:32-n8:16:32-S128"
target triple = "i386-unknown-linux-gnu"

; Function Attrs: noinline nounwind
define i32 @main(i32 %argc, i8** %argv) #0 {
  %1 = load i8** %argv, align 4
  %2 = load i8* %1, align 1
  %3 = icmp slt i8 %2, 0
  br i1 %3, label %4, label %9

; <label>:4                                       ; preds = %0
  %5 = load i8** %argv, align 4
  %6 = load i8* %5, align 1
  %7 = sext i8 %6 to i32
  %8 = sub nsw i32 0, %7
  br label %13

; <label>:9                                       ; preds = %0
  %10 = load i8** %argv, align 4
  %11 = load i8* %10, align 1
  %12 = sext i8 %11 to i32
  br label %13

; <label>:13                                      ; preds = %9, %4
  %14 = phi i32 [ %8, %4 ], [ %12, %9 ]
  %15 = getelementptr inbounds i8** %argv, i32 1
  %16 = load i8** %15, align 4
  %17 = load i8* %16, align 1
  %18 = icmp slt i8 %17, 0
  br i1 %18, label %19, label %25

; <label>:19                                      ; preds = %13
  %20 = getelementptr inbounds i8** %argv, i32 1
  %21 = load i8** %20, align 4
  %22 = load i8* %21, align 1
  %23 = sext i8 %22 to i32
  %24 = sub nsw i32 0, %23
  br label %30

; <label>:25                                      ; preds = %13
  %26 = getelementptr inbounds i8** %argv, i32 1
  %27 = load i8** %26, align 4
  %28 = load i8* %27, align 1
  %29 = sext i8 %28 to i32
  br label %30

; <label>:30                                      ; preds = %25, %19
  %31 = phi i32 [ %24, %19 ], [ %29, %25 ]
  %32 = icmp sgt i32 %14, %31
  br i1 %32, label %33, label %48

; <label>:33                                      ; preds = %30
  %34 = load i8** %argv, align 4
  %35 = load i8* %34, align 1
  %36 = icmp slt i8 %35, 0
  br i1 %36, label %37, label %42

; <label>:37                                      ; preds = %33
  %38 = load i8** %argv, align 4
  %39 = load i8* %38, align 1
  %40 = sext i8 %39 to i32
  %41 = sub nsw i32 0, %40
  br label %46

; <label>:42                                      ; preds = %33
  %43 = load i8** %argv, align 4
  %44 = load i8* %43, align 1
  %45 = sext i8 %44 to i32
  br label %46

; <label>:46                                      ; preds = %42, %37
  %47 = phi i32 [ %41, %37 ], [ %45, %42 ]
  br label %66

; <label>:48                                      ; preds = %30
  %49 = getelementptr inbounds i8** %argv, i32 1
  %50 = load i8** %49, align 4
  %51 = load i8* %50, align 1
  %52 = icmp slt i8 %51, 0
  br i1 %52, label %53, label %59

; <label>:53                                      ; preds = %48
  %54 = getelementptr inbounds i8** %argv, i32 1
  %55 = load i8** %54, align 4
  %56 = load i8* %55, align 1
  %57 = sext i8 %56 to i32
  %58 = sub nsw i32 0, %57
  br label %64

; <label>:59                                      ; preds = %48
  %60 = getelementptr inbounds i8** %argv, i32 1
  %61 = load i8** %60, align 4
  %62 = load i8* %61, align 1
  %63 = sext i8 %62 to i32
  br label %64

; <label>:64                                      ; preds = %59, %53
  %65 = phi i32 [ %58, %53 ], [ %63, %59 ]
  br label %66

; <label>:66                                      ; preds = %64, %46
  %67 = phi i32 [ %47, %46 ], [ %65, %64 ]
  %68 = load i8** %argv, align 4
  %69 = load i8* %68, align 1
  %70 = icmp slt i8 %69, 0
  br i1 %70, label %71, label %76

; <label>:71                                      ; preds = %66
  %72 = load i8** %argv, align 4
  %73 = load i8* %72, align 1
  %74 = sext i8 %73 to i32
  %75 = sub nsw i32 0, %74
  br label %80

; <label>:76                                      ; preds = %66
  %77 = load i8** %argv, align 4
  %78 = load i8* %77, align 1
  %79 = sext i8 %78 to i32
  br label %80

; <label>:80                                      ; preds = %76, %71
  %81 = phi i32 [ %75, %71 ], [ %79, %76 ]
  %82 = getelementptr inbounds i8** %argv, i32 1
  %83 = load i8** %82, align 4
  %84 = load i8* %83, align 1
  %85 = icmp slt i8 %84, 0
  br i1 %85, label %86, label %92

; <label>:86                                      ; preds = %80
  %87 = getelementptr inbounds i8** %argv, i32 1
  %88 = load i8** %87, align 4
  %89 = load i8* %88, align 1
  %90 = sext i8 %89 to i32
  %91 = sub nsw i32 0, %90
  br label %97

; <label>:92                                      ; preds = %80
  %93 = getelementptr inbounds i8** %argv, i32 1
  %94 = load i8** %93, align 4
  %95 = load i8* %94, align 1
  %96 = sext i8 %95 to i32
  br label %97

; <label>:97                                      ; preds = %92, %86
  %98 = phi i32 [ %91, %86 ], [ %96, %92 ]
  %99 = icmp slt i32 %81, %98
  br i1 %99, label %100, label %115

; <label>:100                                     ; preds = %97
  %101 = load i8** %argv, align 4
  %102 = load i8* %101, align 1
  %103 = icmp slt i8 %102, 0
  br i1 %103, label %104, label %109

; <label>:104                                     ; preds = %100
  %105 = load i8** %argv, align 4
  %106 = load i8* %105, align 1
  %107 = sext i8 %106 to i32
  %108 = sub nsw i32 0, %107
  br label %113

; <label>:109                                     ; preds = %100
  %110 = load i8** %argv, align 4
  %111 = load i8* %110, align 1
  %112 = sext i8 %111 to i32
  br label %113

; <label>:113                                     ; preds = %109, %104
  %114 = phi i32 [ %108, %104 ], [ %112, %109 ]
  br label %133

; <label>:115                                     ; preds = %97
  %116 = getelementptr inbounds i8** %argv, i32 1
  %117 = load i8** %116, align 4
  %118 = load i8* %117, align 1
  %119 = icmp slt i8 %118, 0
  br i1 %119, label %120, label %126

; <label>:120                                     ; preds = %115
  %121 = getelementptr inbounds i8** %argv, i32 1
  %122 = load i8** %121, align 4
  %123 = load i8* %122, align 1
  %124 = sext i8 %123 to i32
  %125 = sub nsw i32 0, %124
  br label %131

; <label>:126                                     ; preds = %115
  %127 = getelementptr inbounds i8** %argv, i32 1
  %128 = load i8** %127, align 4
  %129 = load i8* %128, align 1
  %130 = sext i8 %129 to i32
  br label %131

; <label>:131                                     ; preds = %126, %120
  %132 = phi i32 [ %125, %120 ], [ %130, %126 ]
  br label %133

; <label>:133                                     ; preds = %131, %113
  %134 = phi i32 [ %114, %113 ], [ %132, %131 ]
  %135 = ashr i32 %67, 3
  %136 = sub nsw i32 %67, %135
  %137 = ashr i32 %134, 1
  %138 = add nsw i32 %136, %137
  %139 = icmp sgt i32 %67, %138
  br i1 %139, label %140, label %141

; <label>:140                                     ; preds = %133
  br label %146

; <label>:141                                     ; preds = %133
  %142 = ashr i32 %67, 3
  %143 = sub nsw i32 %67, %142
  %144 = ashr i32 %134, 1
  %145 = add nsw i32 %143, %144
  br label %146

; <label>:146                                     ; preds = %141, %140
  %147 = phi i32 [ %67, %140 ], [ %145, %141 ]
  ret i32 %147
}

attributes #0 = { noinline nounwind "less-precise-fpmad"="false" "no-frame-pointer-elim"="true" "no-frame-pointer-elim-non-leaf" "no-infs-fp-math"="false" "no-nans-fp-math"="false" "stack-protector-buffer-size"="8" "unsafe-fp-math"="false" "use-soft-float"="false" }

!llvm.ident = !{!0, !0, !0, !0, !0, !0, !0, !0, !0, !0, !0, !0, !0, !0, !0, !0, !0, !0, !0, !0, !0, !0, !0, !0, !0, !0, !0, !0, !0, !0, !0, !0, !0, !0, !0, !0, !0, !0, !0, !0, !0, !0, !0, !0, !0, !0, !0, !0, !0, !0, !0, !0, !0, !0, !0, !0, !0, !0, !0, !0, !0, !0, !0, !0, !0, !0, !0, !0, !0, !0, !0, !0, !0, !0, !0, !0, !0, !0, !0, !0, !0, !0, !0, !0, !0, !0, !0, !0, !0, !0, !0, !0, !0, !0, !0, !0, !0, !0, !0, !0, !0, !0, !0, !0, !0, !0, !0, !0, !0, !0, !0, !0, !0, !0, !0, !0, !0, !0, !0, !0, !0, !0, !0, !0, !0, !0, !0, !0, !0, !0, !0, !0, !0, !0, !0, !0, !0, !0, !0, !0, !0, !0, !0, !0, !0, !0, !0, !0, !0, !0, !0, !0, !0, !0, !0, !0, !0, !0, !0, !0, !0, !0, !0, !0, !0, !0, !0, !0, !0, !0, !0, !0, !0, !0, !0, !0, !0, !0, !0, !0, !0, !0, !0, !0, !0, !0, !0, !0, !0, !0, !0, !0, !0, !0, !0, !0, !0, !0, !0, !0, !0, !0, !0, !0, !0, !0, !0, !0, !0, !0, !0, !0, !0, !0, !0, !0, !0, !0}

!0 = metadata !{metadata !"clang version 3.5.0 (tags/RELEASE_350/final)"}

\end{lstlisting}